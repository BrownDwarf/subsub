%\documentclass[iop,revtex4]{emulateapj}% change onecolumn to iop for fancy, iop to twocolumn for manuscript
\documentclass[twocolumn]{emulateapj}% change onecolumn to iop for fancy, iop to onecolumn for manuscript
%\documentclass[12pt,preprint]{aastex}

\usepackage{graphicx}
\usepackage[caption=false]{subfig}
%\usepackage{lineno}
%\usepackage{blindtext}
%\linenumbers
%\usepackage{breqn}
\usepackage{amsmath}

\let\pwiflocal=\iffalse \let\pwifjournal=\iffalse
%From: http://arxiv.org/format/1512.00483
%\input{setup}
\input{mgs_setup}


\newcommand{\iancze}{{\sc C15}}

\providecommand{\eprint}[1]{\href{http://arxiv.org/abs/#1}{#1}}
\providecommand{\adsurl}[1]{\href{#1}{ADS}}
\newcommand{\name}{LkCa 4 }
\newcommand{\project}[1]{\textsl{#1}}
%\def\vsini{$v\sin{i_*}$}

\slugcomment{In preparation}

\shorttitle{Spots on Sub-subgiants}

\shortauthors{TBD}

\bibliographystyle{yahapj}

\begin{document}

\title{Starspots on sub sub giants }

\author{TBD,\altaffilmark{1}, author list TBD}


\altaffiltext{1}{TBD}

\begin{abstract}

We measure the starspots on a sub sub giant.

\end{abstract}

\keywords{stars: fundamental parameters ---  stars: statistics}

\maketitle

\section{Introduction}\label{sec:intro}

We do not know what causes some sub-sub giants!
But now we know a little... more. Hence this paper!

Sub-subgiant stars are stars that lie below the subgiant branch on a cluster optical color-magnitude diagram (CMD), but are too red to be main sequence stars. These stars are commonly found in evolved open clusters and globular clusters, with 65 sub-subgiants currently identified across 17 different clusters \citep{geller17}. The majority of sub-subgiant stars are single-lined spectroscopic binaries with short orbital periods of a few days with moderate X-ray luminosities of $10^{30}$ to $10^{31}$ erg s$^{-1}$ \citep[and references therein]{geller17}.

The presence of sub-subgiants cannot be explained with typical single-star stellar evolutionary pathways. Three possible formation scenarios for sub-subgiants were put forth by \citet{leiner17}: mass transfer in a binary system, stripping of a subgiant star envelope through a dynamical encounter, and reduced luminosity as a result of inhibited convection due to the presence of strong magnetic fields. \citet{leiner17} conclude that strong magnetic fields due to binary interactions would produce the largest number of sub-subgiants.

In this paper we focus on a single sub-subgiant system, S1063 in the open cluster M67. This system is a prototypical sub-subgiant star, with an single-lined spectroscopic orbital period of 18.4 days \citep{geller17}, an X-ray luminosity of $1.3\times10^{31}$ erg s$^{-1}$ \citep{vandenberg99}, and a variable light curve in \textit{K2} (REF LIGHT CURVE SECTION?).

\subsection{Starspots as confounding factors}
Mass, age, and metallicity uniquely map a main sequence star to its HR diagram position.  A fourth factor---rotation---confounds this mapping in as-yet-unknown ways.  Rotation might enhance spread in HR diagram positions (cite XX Davenport, Douglas?), with the most conspicuous spreads in the pre-main sequence regime (cite XX Covey, Stauffer).  Spreads in the HR diagram offer clues to the consequences of rotation.  Increased rotation heightens the magnetic dynamo strength and concommitant surface magnetic field.  These magnetic fields suppress convective efficiency, meaning the star must increase in size at a lower effective temperature to allow the same amount of internal energy to escape (cite XX Feiden): rotating stars become bigger and cooler than their non-rotating counterparts (cite XX Somers).  The interplay of rotation, dynamo, and surface fields remain an active area of research, with bright prospects for a unified theory involving the degree of magnetic complexity parameter (cite XX Garraffo).

Surface magnetic fields offer two key observational manifestations.  The Zeeman Effect splits spectral line levels in magnetic-sensitive atomic transitions (cite XX Johns-Krull).  Starspots induce stellar surface inhomogeneities that can be seen in the modulation of

The story in the post-main sequence is less clear.  Angular momentum transport governs rotation  as stars evolve over orders of magnitude in size.


%Fill in more theory here
\citet{somers15} emphasized the roll of starspots in inferring stellar ages.

\begin{itemize}
\item What is a sub-subgiant?
  \begin{itemize}
  \item What causes sub sub giant stars (Leiner et al. 2017)
  \end{itemize}
\item M67 S1063
\begin{itemize}
  \item Prototypical subsub (Geller et al. 2017)
  \item Binary orbit, SB1
\end{itemize}
\item Starspots as confounding factors
\begin{itemize}
  \item Inhibit convective efficiency (redder and bigger)
  \item Also confound observations: assign incorrect Teff
  \item This paper aims to measure the starspot coverage and temp
\end{itemize}
\item Layout of this paper
\end{itemize}

\section{Observation and data reduction}
\begin{itemize}
\item IGRINS observations
\item Ground-based photometric monitoring- ASASSN, AAVSO, ASAS
\item K2 data (Campaign 5), *stretch goal* C16
\item *stretch goal* APOGEE?
\item Gaia data (membership confirmed)
\end{itemize}

\section{Analysis}
\begin{itemize}
\item Summary and assumptions of our methods
\item K2 Superstamp(s)
\begin{itemize}
  \item K2 detrending
  \item Interpreting lightcurves
  \item Period and amplitude of lightcurve, with multi-term Lomb-Scargle + Fourier reconstruction
\end{itemize}
\item Phase folded archival V-band photometry (ASASSN+)
\subsection{IGRINS two-temperature spectral inference w/ Starfish}

We performed two-temperature probabilistic spectral decomposition on the IGRINS $H-$band spectrum.  We applied the spectral inference framework \texttt{Starfish} \citep{czekala15}, recently extended to support composite spectra comprised of mixtures of two distinct photospheric components \citep{gullysantiago17}.  Here, the two temperature components are labeled as $T_{\mathrm{spot}}$ and $T_{\mathrm{amb}}$ for the starspot and ambient photospheric emission respectively, with a filling factor $f$ defined as the ratio of collective projected surface area of the spot groups to the projected area of the star.

We employed the \texttt{PHOENIX} precomputed synthetic model grid with grid ranges of $3000 < T_{\mathrm{eff}} \; (K) < 5300 $, $3 < \log{g \;(cm/s)}  < 4 $, and $ -0.5 <  [\mathrm{Fe}/\mathrm{H}] <0.5$.  We trained the spectral emulator \citep{czekala15} on this grid range, while preserving the absolute model mean fluxes to enable accurate flux comparison between two photospheres of disparate characteristic temperatures.  This new approach offers improved accuracy over the scalar flux interpolated approach introduced in Appendix XX of \citet{gullysantiago17}, especially for such a large dynamic range in effective temperature.  The spectral emulator approach propagates the uncertainty attributable to the coarsely sampled \texttt{PHOENIX} models.

The pre-defined grid ranges place uniform priors over their domain.  Additionally, a threshold of 4500 K separated the allowed domains for the spot and ambient temperatures, yielding uniform priors $3000 < T_{\mathrm{spot}} \; (K) < 4500 $ and $4500 < T_{\mathrm{amb}} \; (K) < 5300$.

\subsubsection{MCMC convergence and posterior predictive checks}
Each IGRINS spectral order was fit independently, yielding over 20 individual sets of MCMC posteriors.  We employed \texttt{emcee} \citep{foreman13} with 5000 samples and 40 walkers, spot-checking the MCMC chains for signatures of steady-state posterior probability distributions suggestive of convergence.  Some orders did not pass our convergence criteria, usually due to poor initialization of nuissance parameters or overfitting.  These specrtral orders were removed from future analysis, yielding a total of nine spectral orders, shown in Figure \ref{fig:IGRINS_spectra3x3}.

\subsubsection{FIGURE: IGRINS Spectra}
\begin{figure*}
 \centering
 \includegraphics[width=0.98\textwidth]{figures/H_band_spectra_3x3.pdf}
 \caption{Nine $H-$band IGRINS spectral orders with probabilistic spectral decomposition.}
 \label{fig:IGRINS_spectra3x3}
\end{figure*}

\subsubsection{Internal consistency of vsini, $v_z$}
We additionally spot-checked the MCMC posteriors with posterior predictive checks... XX

\item Analysis of near-IR flux contribution from binary companion
\begin{itemize}
  \item Limits on companion types
\end{itemize}
\end{itemize}

\section{Results}

Using the \texttt{Starfish} spectral inference results we investigate the relationship between spot temperature and filling factor. In Figure~\ref{fig:tspot_fillingfactor3x3} we show 2-dimensional distributions of filling factor and spot temperature of the last 200 samples for the nine orders with accepted fits. Similar trends between spot temperature and filling factor appear across most of the orders. Across all nine orders, the median filling factor value is 32\% with a standard deviation of 7\%, with a corresponding spot temperature of $4000 \pm 200$ K. The ambient photosphere temperature associated with this spot signature is $5200\pm25$ K. This is similar to the optical spectroscopic temperature of 5000 K determined by \citet{mathieu03}, which is expected as the optical spectrum will have less significant spot signatures than the IGRINS spectrum in the NIR.

The presence of spots impacts the effective temperature of S1063. Taking into account both temperature components, we calculate the effective temperature using
\begin{equation}
T_{\textrm{eff}}^4 = f_{\textrm{spot}} T_{\textrm{spot}}^4 + (1 -f_{\textrm{spot}}) T_{\textrm{ambient}}^4 .
\end{equation}
This results in an updated effective temperature for S1063 of $4900\pm125$ K.

Over the course of \textit{K2} C5, the IGRINS spectrum was observed at approximately 97\% of maximum light (REF LIGHT CURVE FIGURE). From a first-order interpretation of light curve amplitudes, minimum light corresponds to the largest star spot coverage while maximum light corresponds to the least star spot coverage, although these assumptions are complicated by the possibility of spot evolution \citep{basri18}. At the time of observation, these spectral inference results suggest that the lower limit on the total star spot coverage of S1063 is approximately 32\%. We note that, in this case, the light curve maximum is not consistent with zero spot coverage. As seen in Figure LIGHT CURVE, during \textit{K2} C5 the absolute light curve minimum (occuring at HJD **some date here**) was approximately 5\% lower than the flux at the time of observation. This suggests that the maximum spot filling factor during the time of C5 was closer to 40\%.



\begin{itemize}
\item We detect spots in spectra
\begin{itemize}
  \item S1063 has $32 \pm 7$\% coverage fraction of spots with Tspot $4000\pm200$ K based on IGRINS + Starfish
  \item Revised effective temperature using both temperature components $[f_{\textrm{spot}} * T_{\textrm{spot}}^4 + (1 -f_{\textrm{spot}}) * T_{\textrm{ambient}}^4] = T_{\textrm{eff}}^4$

  \item *bonus* Rsini
\end{itemize}
\item IGRINS observations occurred at maximum of lightcurve, so total spot coverage is even greater
\item *bonus* total spot filling factor estimate given light curve magnitude
\item FIGURE: $T_{spot}$ versus $f_{spot}$ plot


 \begin{figure*}[h]
   \centering
   \begin{tabular}{ccc}
     \subfloat{\includegraphics[width=2in]{figures/H_band_Tspot_fillingfactor_m119.pdf}} &
     \subfloat{\includegraphics[width=2in]{figures/H_band_Tspot_fillingfactor_m118.pdf}} &
     \subfloat{\includegraphics[width=2in]{figures/H_band_Tspot_fillingfactor_m116.pdf}} \\
     \subfloat{\includegraphics[width=2in]{figures/H_band_Tspot_fillingfactor_m114.pdf}} &
     \subfloat{\includegraphics[width=2in]{figures/H_band_Tspot_fillingfactor_m113.pdf}} &
     \subfloat{\includegraphics[width=2in]{figures/H_band_Tspot_fillingfactor_m110.pdf}} \\
     \subfloat{\includegraphics[width=2in]{figures/H_band_Tspot_fillingfactor_m109.pdf}} &
     \subfloat{\includegraphics[width=2in]{figures/H_band_Tspot_fillingfactor_m107.pdf}} &
     \subfloat{\includegraphics[width=2in]{figures/H_band_Tspot_fillingfactor_m106.pdf}}
   \end{tabular}
 \caption{2-dimensional distributions of filling factor and spot temperature for the nine accepted IGRINS orders for S1063. The median filling factor across these nine orders is $32 \pm 7$\% with a spot temperature of $4000\pm200$ K. As the IGRINS spectrum was observed near maximum light, these show a lower limit on the total spot coverage fraction. }
 \label{fig:tspot_fillingfactor3x3}
 \end{figure*}

\item What coverage fraction would we have measured across the rotational phase?
\end{itemize}

\section{Discussion}


The spot coverage fraction measured here is consistent with the range of spot coverage seen on RS CVn of 30--40\% from measuring TiO band strength \citep{oneal96, oneal98, oneal04} and Doppler imaging \citep{hackman12}. The similar spot coverage fraction on S1063 is further evidence that this sub-subgiant and likely other sub-subgiants have high magnetic activity.

In the absence of this spectral inference technique one could assume the \textit{K2} C5 light curve maximum corresponded to zero spot presence with the light curve amplitude change implying a maximum spot coverage of 7--10\%. This provides a fundamentally different view of the star ...

\begin{itemize}
\item Spot coverage is consistent with formation theories
\item Conceivable geometries with circumpolar active longitudes, or migrating active latitudes
\item Biases introduced if we assume a spot-free model
\begin{itemize}
  \item Where does subsub sit in a new HR diagram? (new Somers models)
  \item *bonus* FIGURE: PMS HR diagram with new Somers tracks
  \item Spot impact on SED fits
\end{itemize}
\end{itemize}

\section{Conclusions}

Reiteration here.

\clearpage
\pagebreak


\appendix

\section{Are starspots confusing?}
\label{methods-details}

Short answer: no!

\acknowledgements

%ADS
We thank ADS!

%Kepler
This paper includes data collected by the Kepler mission. Funding for the Kepler mission is provided by the NASA Science Mission directorate.

% MAST
Some/all of the data presented in this paper were obtained from the Mikulski Archive for Space Telescopes (MAST). STScI is operated by the Association of Universities for Research in Astronomy, Inc., under NASA contract NAS5-26555.


{\it Facilities:} \facility{Smith (IGRINS)}, \facility{AAVSO}, \facility{INTEGRAL (OMC)}, \facility{ASAS}, \facility{Gaia}

{\it Software: }
 \project{pandas} \citep{mckinney10},
 \project{emcee} \citep{foreman13},
 \project{matplotlib} \citep{hunter07},
 \project{numpy} \citep{vanderwalt11},
 \project{scipy} \citep{jones01},
 \project{ipython} \citep{perez07},
 \project{gatspy} \citep{JakeVanderplas2015},
 \project{starfish} \citep{czekala15},
 \project{seaborn} \citep{waskom14}
%\software{%
% \project{pandas} \citep{mckinney10}
%    \project{emcee} \citep{foreman13},
% \project{matplotlib} \citep{hunter07},
% \project{numpy} \citep{vanderwalt11},
% \project{scipy} \citep{jones01},
% \project{ipython} \citep{perez07},
% \project{gatspy} \citep{JakeVanderplas2015},
% \project{starfish} \citep{czekala15}}.

\clearpage

\bibliographystyle{apj}
\bibliography{ms}

\end{document}

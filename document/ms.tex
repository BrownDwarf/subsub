%\documentclass[iop,revtex4]{emulateapj}% change onecolumn to iop for fancy, iop to twocolumn for manuscript
\documentclass[twocolumn]{emulateapj}% change onecolumn to iop for fancy, iop to onecolumn for manuscript
%\documentclass[12pt,preprint]{aastex}

%\usepackage{lineno}
%\usepackage{blindtext}
%\linenumbers
%\usepackage{breqn}
\usepackage{amsmath}

\let\pwiflocal=\iffalse \let\pwifjournal=\iffalse
%From: http://arxiv.org/format/1512.00483
%\input{setup}
\usepackage{enumerate}
\usepackage{amsmath,amssymb}
\usepackage{bm}
\usepackage{color}
\usepackage[utf8]{inputenc}

%% For anyone who downloaded my source file from arxiv:
%% I stole most of this setup.tex from a paper by Peter .K.G. Williams, but I made a bunch of edits to satisfy my own needs. You might check his paper out (http://arxiv.org/abs/1409.4411) for the original source file or contact him if you have any questions, since I don't really understand how some of these things work. 
%One cool thing it does is you can define an object, just that when someone clicks on the pdf it will link to simbad. I could never quite get this to work, probably because you have to get the text exactly right and my motivation for getting it to work was not super high. 


% basic packages
\usepackage{amsmath,amssymb}
\usepackage[breaklinks,colorlinks,urlcolor=blue,citecolor=blue,linkcolor=blue]{hyperref}
\usepackage{epsfig}    
\usepackage{graphicx}    
\usepackage{lineno}
\usepackage{natbib}
\usepackage{bigints}
\usepackage[outdir=./]{epstopdf}



% font stuff
\usepackage[T1]{fontenc}
\pwifjournal\else
  \usepackage{microtype}
\fi


% emulateapj has overly conservative figure widths, I think because some
% people's figures don't have good margins. Override.
\pwifjournal\else
  \makeatletter
  \renewcommand\plotone[1]{%
    \centering \leavevmode \setlength{\plot@width}{0.95\linewidth}
    \includegraphics[width={\eps@scaling\plot@width}]{#1}%
  }%
  \makeatother
\fi


\makeatletter

\newcommand\@simpfx{http://simbad.u-strasbg.fr/simbad/sim-id?Ident=}

\newcommand\MakeObj[4][\@empty]{% [shortname]{ident}{url-escaped}{formalname}
  \pwifjournal%
    \expandafter\newcommand\csname pkgwobj@c@#2\endcsname[1]{\protect\object[#4]{##1}}%
  \else%
    \expandafter\newcommand\csname pkgwobj@c@#2\endcsname[1]{\href{\@simpfx #3}{##1}}%
  \fi%
  \expandafter\newcommand\csname pkgwobj@f#2\endcsname{#4}%
  \ifx\@empty#1%
    \expandafter\newcommand\csname pkgwobj@s#2\endcsname{#4}%
  \else%
    \expandafter\newcommand\csname pkgwobj@s#2\endcsname{#1}%
  \fi}%

\newcommand\MakeTrunc[2]{% {ident}{truncname}
  \expandafter\newcommand\csname pkgwobj@t#1\endcsname{#2}}%

\newcommand{\obj}[1]{%
  \expandafter\ifx\csname pkgwobj@c@#1\endcsname\relax%
    \textbf{[unknown object!]}%
  \else%
    \csname pkgwobj@c@#1\endcsname{\csname pkgwobj@s#1\endcsname}%
  \fi}
\newcommand{\objf}[1]{%
  \expandafter\ifx\csname pkgwobj@c@#1\endcsname\relax%
    \textbf{[unknown object!]}%
  \else%
    \csname pkgwobj@c@#1\endcsname{\csname pkgwobj@f#1\endcsname}%
  \fi}
\newcommand{\objt}[1]{%
  \expandafter\ifx\csname pkgwobj@c@#1\endcsname\relax%
    \textbf{[unknown object!]}%
  \else%
    \csname pkgwobj@c@#1\endcsname{\csname pkgwobj@t#1\endcsname}%
  \fi}

\makeatother


% Evil magic to patch natbib to only highlight the year paper refs, not the
% authors too; as seen in ApJ. From
% http://tex.stackexchange.com/questions/23227/.

\pwifjournal\else
  \usepackage{etoolbox}
  \makeatletter
  \patchcmd{\NAT@citex}
    {\@citea\NAT@hyper@{%
       \NAT@nmfmt{\NAT@nm}%
       \hyper@natlinkbreak{\NAT@aysep\NAT@spacechar}{\@citeb\@extra@b@citeb}%
       \NAT@date}}
    {\@citea\NAT@nmfmt{\NAT@nm}%
     \NAT@aysep\NAT@spacechar\NAT@hyper@{\NAT@date}}{}{}
  \patchcmd{\NAT@citex}
    {\@citea\NAT@hyper@{%
       \NAT@nmfmt{\NAT@nm}%
       \hyper@natlinkbreak{\NAT@spacechar\NAT@@open\if*#1*\else#1\NAT@spacechar\fi}%
         {\@citeb\@extra@b@citeb}%
       \NAT@date}}
    {\@citea\NAT@nmfmt{\NAT@nm}%
     \NAT@spacechar\NAT@@open\if*#1*\else#1\NAT@spacechar\fi\NAT@hyper@{\NAT@date}}
    {}{}
  \makeatother
\fi

\newcommand{\prob}{{\rm prob}}
\newcommand{\qN}{\{q_i\}_{i=1}^N}
\newcommand{\qM}{\{q_{im}\}_{i=1,m=0}^{N,M}}
\newcommand{\yN}{\{y_i\}_{i=1}^N}

\newcommand{\kms}{ \textrm{km s}^{-1} }

\newcommand{\vM}{\mathsf{M}}
\newcommand{\vD}{\mathsf{D}}
\newcommand{\vR}{\mathsf{R}}
\newcommand{\vC}{\mathsf{C}}
\newcommand{\fM}{ \vec{{\bm M}}}
\newcommand{\fMi}{M_i}
\newcommand{\fD}{ \vec{{\bm D}}}
\newcommand{\fDi}{D_i}
\newcommand{\fR}{ {\bm R}}
\newcommand{\dd}{\,{\rm d}}
\newcommand{\trans}{\mathsf{T}}
\newcommand{\teff}{T_\textrm{eff}}
\newcommand{\teffa}{T_\textrm{amb}}
\newcommand{\teffb}{T_\textrm{spot}}
\newcommand{\logg}{\log g}
\newcommand{\Z}{[{\rm Fe}/{\rm H}]}
\newcommand{\A}{[\alpha/{\rm Fe}]}
\newcommand{\vsini}{v \sin i}
\newcommand{\matern}{Mat\'{e}rn}
\newcommand{\HK}{$\textrm{H}_2$O-K2}
\newcommand{\cc}[2]{c_{#2}^{(#1)}} 

\newcommand{\flam}{f_\lambda}
\newcommand{\vt}{ {\bm \theta}}
\newcommand{\vT}{ {\bm \Theta}}
\newcommand{\vp}{ {\bm \phi}}
\newcommand{\vP}{ {\bm \Phi}}
\newcommand{\cheb}{ \vp_{\mathsf{P}}}
\newcommand{\chebi}[1]{ \vp_{\textrm{Cheb}_{#1}}}
\newcommand{\Cheb}{ \vP_{\textrm{Cheb}}}
\newcommand{\Chebi}[1]{ \vP_{\textrm{Cheb}_{\ne #1}}} 
\newcommand{\cov}{ \vp_{\mathsf{C}}}
\newcommand{\covi}[1]{ \vp_{\textrm{cov}_{#1}}} 
\newcommand{\Cov}{ \vP_{\textrm{cov}}}
\newcommand{\Covi}[1]{ \vP_{\textrm{cov}_{\ne #1}}} 

\newcommand{\allParameters}{\vT} 
\newcommand{\nuisanceParameters}{\vP} 

\newcommand{\KK}{\mathcal{K}}
\newcommand{\Kglobal}{\KK^{\textrm{G}}}
\newcommand{\Klocal}{\KK^{\textrm{L}}}

\newcommand{\Gl}{Gl\,51}
\newcommand{\PHOENIX}{{\sc Phoenix}}

% Appendix commands
\newcommand{\wg}{\mathbf{w}^\textrm{grid}}
\newcommand{\wgh}{\hat{\mathbf{w}}^\textrm{grid}}

\newcommand{\Sg}{\mathbf{\Sigma}^\textrm{grid}}


\newcommand{\todo}[1]{ \textcolor{blue}{\\TODO: #1}}
\newcommand{\comm}[1]{ \textcolor{red}{MGS: #1}}
\newcommand{\hili}[1]{ \textcolor{green}{#1}}
\newcommand{\ctext}[1]{ \textcolor{blue}{\% #1}}


%  From Peter Williams and Andy Mann again:
\newcommand{\um}{$\mu$m}


\newcommand{\iancze}{{\sc C15}}

\providecommand{\eprint}[1]{\href{http://arxiv.org/abs/#1}{#1}}
\providecommand{\adsurl}[1]{\href{#1}{ADS}}
\newcommand{\name}{LkCa 4 }
\newcommand{\project}[1]{\textsl{#1}}
%\def\vsini{$v\sin{i_*}$}


\slugcomment{In preparation}

\shorttitle{Spots on Sub-subgiants}

\shortauthors{TBD}

\bibliographystyle{yahapj}

\begin{document}

\title{Starspots on sub sub giants }

\author{TBD,\altaffilmark{1}, author list TBD}


\altaffiltext{1}{TBD}

\begin{abstract}

We measure the starspots on a sub sub giant.

\end{abstract}

\keywords{stars: fundamental parameters ---  stars: statistics}

\maketitle

\section{Introduction}\label{sec:intro}

We do not know what causes some sub-sub giants!
But now we know a little... more. Hence this paper!

%Fill in more theory here
\citet{somers15} emphasized the roll of starspots in inferring stellar ages.

\begin{itemize}
\item What is a sub-subgiant?
  \begin{itemize}
  \item What causes sub sub giant stars (Leiner et al. 2017)
  \end{itemize}
\item M67 S1063
\begin{itemize}
  \item Prototypical subsub (Geller et al. 2017)
  \item Binary orbit, SB1
\end{itemize}
\item Starspots as confounding factors
\begin{itemize}
  \item Inhibit convective efficiency (redder and bigger)
  \item Also confound observations: assign incorrect Teff
  \item This paper aims to measure the starspot coverage and temp
\end{itemize}
\item Layout of this paper
\end{itemize}

\section{Observation and data reduction}
\begin{itemize}
\item IGRINS observations
\item Ground-based photometric monitoring- ASASSN, AAVSO, ASAS
\item K2 data (Campaign 5), *stretch goal* C16
\item *stretch goal* APOGEE?
\item Gaia data (membership confirmed)
\end{itemize}

\section{Analysis}
\begin{itemize}
\item Summary and assumptions of our methods
\item K2 Superstamp(s)
\begin{itemize}
  \item K2 detrending
  \item Interpreting lightcurves
  \item Period and amplitude of lightcurve, with multi-term Lomb-Scargle + Fourier reconstruction
\end{itemize}
\item Phase folded archival V-band photometry (ASASSN+)
\item IGRINS two-temperature spectral inference w/ Starfish
\begin{itemize}
  \item MCMC convergence and posterior predictive checks
  \item FIGURE: IGRINS Spectra
  \item Internal consistency of vsini, $v_z$
  \item FIGURE: Violin plot
\end{itemize}
\item Analysis of near-IR flux contribution from binary companion
\begin{itemize}
  \item Limits on companion types
\end{itemize}
\end{itemize}

\section{Results}
\begin{itemize}
\item We detect spots in spectra
\begin{itemize}
  \item S1063 has ~X\% coverage fraction of spots with Tspot ~ Y based on IGRINS + Starfish
  \item *bonus* Revised effective temperature using both temperature components $[f_{\textrm{spot}} * T_{\textrm{spot}}^4 + (1 -f_{\textrm{spot}}) * T_{\textrm{ambient}}^4] = T_{\textrm{eff}}^4$
  \item *bonus* Rsini
\end{itemize}
\item IGRINS observations occurred at maximum of lightcurve, so total spot coverage is even greater
\item FIGURE: $T_{spot}$ versus $f_{spot}$ plot
\item What coverage fraction would we have measured across the rotational phase?
\end{itemize}

\section{Discussion}
\begin{itemize}
\item Spot coverage is consistent with formation theories
\item Conceivable geometries with circumpolar active longitudes, or migrating active latitudes
\item Biases introduced if we assume a spot-free model
\begin{itemize}
  \item Where does subsub sit in a new HR diagram? (new Somers models)
  \item *bonus* FIGURE: PMS HR diagram with new Somers tracks
  \item Spot impact on SED fits
\end{itemize}
\end{itemize}

\section{Conclusions}

Reiteration here.

\clearpage
\pagebreak


\appendix

\section{Are starspots confusing?}
\label{methods-details}

Short answer: no!

\acknowledgements

%ADS
We thank ADS!

%Kepler
This paper includes data collected by the Kepler mission. Funding for the Kepler mission is provided by the NASA Science Mission directorate.

% MAST
Some/all of the data presented in this paper were obtained from the Mikulski Archive for Space Telescopes (MAST). STScI is operated by the Association of Universities for Research in Astronomy, Inc., under NASA contract NAS5-26555.


{\it Facilities:} \facility{Smith (IGRINS)}, \facility{AAVSO}, \facility{INTEGRAL (OMC)}, \facility{ASAS}, \facility{Gaia}

{\it Software: }
 \project{pandas} \citep{mckinney10},
 \project{emcee} \citep{foreman13},
 \project{matplotlib} \citep{hunter07},
 \project{numpy} \citep{vanderwalt11},
 \project{scipy} \citep{jones01},
 \project{ipython} \citep{perez07},
 \project{gatspy} \citep{JakeVanderplas2015},
 \project{starfish} \citep{czekala15},
 \project{seaborn} \citep{waskom14}
%\software{%
% \project{pandas} \citep{mckinney10}
%    \project{emcee} \citep{foreman13},
% \project{matplotlib} \citep{hunter07},
% \project{numpy} \citep{vanderwalt11},
% \project{scipy} \citep{jones01},
% \project{ipython} \citep{perez07},
% \project{gatspy} \citep{JakeVanderplas2015},
% \project{starfish} \citep{czekala15}}.

\clearpage

\bibliographystyle{apj}
\bibliography{ms}

\end{document}
